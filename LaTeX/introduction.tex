\chapter{前言}
\renewcommand{\baselinestretch}{10} %設定行距
\section{研究動機}
\par
\renewcommand{\baselinestretch}{2} %設定行距
\twelve \qquad 我們是以教學與研究歷程的重要性作為研究動機,因為我們發現很多學術上的研究流程不夠詳細,導致後者無法完全得知這些專案的過程是如何進行,造成他們無法順利地接手,甚至重頭開始進行,所以我們希望不只提供學生在學習與研究流程能夠不只留下具體成果,也能有效呈現更細部的歷程與資訊,以作為學習與研究更有力的佐證資料。
\par
\renewcommand{\baselinestretch}{20} %設定行距
\section{製作目的}
\par
\renewcommand{\baselinestretch}{1} %設定行距
\twelve \qquad 目的分為兩部分:
\begin{enumerate}
\item 探討如何利用 Fossil SCM 虛擬與實體伺服器,讓五專精密機械工程科所有相關師生包含已經畢業的校友,得以透過 @gmail 帳號登入,並在網際內容管理系統中進行知識管理與互動,擬藉此提升課程教學與專題研究效益。
\item 允許使用者透過學校配發的 @gm 登入後,有權限在伺服器上自行建立獨立的倉儲系統並且自行管理。
\end{enumerate}
\par
\renewcommand{\baselinestretch}{20} %設定行距
\section{未來展望}
\par
\renewcommand{\baselinestretch}{1} %設定行距
\twelve \qquad 此專題希望用戶能利用架設的分散式版次管理系統在各別的倉儲或社群進行社會化共同項目開發與資訊交流等,包括允許使用者追蹤其他使用者、組織、軟體庫的相關資訊,對開發項目進行評論且改動等。
\par}