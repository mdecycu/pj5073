\setcounter{chapter}{0}
\chapter{前言}
\pagenumbering{arabic}%數字頁數
\setcounter{page}{1}  %設定頁數
\renewcommand{\baselinestretch}{10} %設定行距
\section{研究動機}
\par
\renewcommand{\baselinestretch}{1} %設定行距
\twelve 本科配合教育部107年重啟五專招生,培養學生一技之長並讓學生從實務應用的層面了解自身之所學、產業的需求、以及未來就業可預期的多元的教育與訓練。透過技術培養及實務經驗累積,從學識與實務面出發,結合業界資源,學習產業精密機械之學識與技能,同時也能學以致用進入職場,使得學校訓練與業界需求能緊密結合,強化我國之國際競爭力學用落差,促進台灣機械工業技術的提升與創新。\\
\par
\renewcommand{\baselinestretch}{1} %設定行距
\twelve 然而,在虎尾科技大學中,精密機械工程科是為重啟五專招生的第一個科,作為第一屆的領頭者,我們不僅需要開創新的選擇,甚至需要為後來新進的成員做鋪墊。剛進到學校時我們對於職涯規劃、修課內容和科上資源都不太清楚。在精密機械科上資訊的執行層面上存在著許多問題,導致學生資訊交流的機會遭遇瓶頸,例如:無法得知先前製作專案的詳細流程,造成他們無法順利地接手,甚至重頭開始進行,所以我們希望不只提供學生在學習與研究流程能夠不只留下具體成果,也能有效呈現更細部的歷程與資訊,以作為學習與研究更有力的佐證資料。\\
\par
\renewcommand{\baselinestretch}{1} %設定行距
\twelve 網路上論壇紛雜,雖然許多學生已熟悉其他論壇的操作,但是我們的論壇專為精密科學生所設置,是一個非常貼近精密機械科學生日常問題和校園生活,例如: 教授擅長的領域、學長姐修課後的資訊、實習公司資訊、專案詳細的製作流程、個人學習經歷等均納入其中,因此專題研究期望以「網際內容管理系統」的設計概念,融入教學型入口網站更貼近學生的服務項目,引導學生有興趣、有需要、而且意願性地利用此網站進行交流,並共同建立一個內容豐富、管理簡便與學生協同經營的「精密機械科學習論壇」。
\par

\renewcommand{\baselinestretch}{20} %設定行距
\section{研究目的}
\par
\renewcommand{\baselinestretch}{1} %設定行距
\twelve 現今網際網路的普及與網路快速傳輸和散布的特性,使得資訊的傳遞與獲取變得更加容易,只要輸入關鍵字就能取得龐大且豐富的相關資訊。網路是一座包羅萬象的資料庫,每一個人都可能是資訊的發布者和接收者,在沒有全面把關的情況下,網路上也充斥著未經查證的「網路謠言、假資訊」,因此讓學生得知正確資訊是非常重要的。因應生活在數位化時代中的我們,使用安全網站搜集正確資訊是非常重要的,而我們先從發展「精密科學習論壇」做為推動目標,除了學習資源的推動、學生協同經營外,如何將精密科透過網際網路加以具體呈現並收宣傳之效,並且能提供精密科歷年的完整資訊。\\
\par
\renewcommand{\baselinestretch}{1} %設定行距
\twelve 本研究師生以精密科上本位發展為出發點,希望能夠將科上課程及本位特色結合全體師生的合作共同營造一個「精密科學習論壇」,目的分為三部分:
\begin{enumerate}
	\item 採取「內容管理系統」的設計概念並且結合「教學應用」與「研究應用」的設計理念,提出「精密科學習論壇」的概念。
	\item 建立一個精密科上學生資源中心的雛形系統,利用 Fossil SCM 虛擬與實體伺服器,讓五專精密機械工程科所有相關師生包含已經畢業的校友,得以透過 @gmail 帳號登入,進行知識管理與互動,擬藉此提升課程教學與專題研究效益,並在內容系統長期使用與管理下,可以當後續學生的資訊來源及研究之參考平台。
	\item 以開放原始碼(Open Source)允許使用者透過學校配發的 @gm 登入後,有權限在伺服器上自行建立獨立的倉儲系統並且自行管理。
\end{enumerate}
\par

\renewcommand{\baselinestretch}{20} %設定行距
\section{技術說明}
\par
\renewcommand{\baselinestretch}{1} %設定行距
\twelve 我們使用的 Fossil SCM 是一個跨平台伺服器,可以執行於 Linux、Windows 等多種平台。它特點是分散式版本控制、問題跟蹤、wiki 服務和部落格。該軟體有一個內建的網路介面,這降低了專案跟蹤的複雜性,並提升了狀態意識。使用者可以簡單地鍵入「fossil ui」,Fossil 就會自動在使用者的網頁瀏覽器中打開一個網頁,提供詳細歷史和狀態資訊。因為是分散式系統架構,所以 Fossil 不需要中央伺服器,儘管使用中央伺服器可以使協同運作變得更容易。
\par

\renewcommand{\baselinestretch}{20} %設定行距
\section{未來展望}
\par
\renewcommand{\baselinestretch}{1} %設定行距
\twelve 此專題希望用戶能利用架設的分散式系統在各別的倉儲或社群進行社會化共同項目開發與資訊交流等,包括允許使用者追蹤其他使用者、組織、軟體庫的相關資訊,對開發項目進行評論且改動等,並且利用版次管理將相關資訊但不同時間點以版本控制呈現,而達到證明歷程,並且能加以分析或探討。
\par}
