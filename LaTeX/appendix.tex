\chapter*{附錄}
\addcontentsline{toc}{chapter}{\sixteen 附錄}
\renewcommand{\baselinestretch}{10} %設定行距
%\section{前言}
\par
\renewcommand{\baselinestretch}{1} %設定行距
\twelve
\par LaTeX為一種程式語言,是一套文件排版系統,可以用來編輯要出版的研究論文、書籍等文件,甚至也可以拿來製作簡報用的投影片,不過與一般程式語言不同的是,它可以直接透過副檔名.tex的文件做排版結構,相較於與程式語言Markdown更較為複雜,但格式若已經成形,後續的步驟相對簡易許多且製作數學公式和排版也相較精準、正確。
\\
\par
\renewcommand{\baselinestretch}{1.7} %設定行距
\begin{center}
\begin{tabular}[c]{|c|c|c|c|c|c|} %表格
\hline
		&相容性&直關性&文件排版&數學公式&細部微調
\\
\hline
LaTeX&$\surd$&		&$\surd$&$\surd$&$\surd$
\\
\hline
Word&		&$\surd$&		&		&$\surd$
\\
\hline
\end{tabular}
\end{center}
\par

\par
\renewcommand{\baselinestretch}{1} %設定行距
\twelve \quad\\LaTex優點:
\par
\begin{enumerate}
	\item \textbf{各種的模組套件:} 除了陽春版的LaTeX,它還有數百種的模組套件(package)提供各種新功能,目前幾乎任何想像得到的功能都有模組套件提供支援,從呈現演算法(algorithm)到呈現棋譜的可說是一應俱全,要用Word做這些事可就困難了。
	\item \textbf{相容性:} 基本上Word是最不跟別人相容的,但因為市場佔有率太高,反而變成大家要去跟它相容。然而,它又採用自己的封閉格式,造成大家沒辦法真正去跟它相容。或許是為了公司利益,但是我們多數的人都是使用者,應該要體認到這是不合理的事情。關於這一方面,LaTeX就更勝一籌,不僅不用擔心裝置上的問題,更重要的是它是一個開源的軟體。
	\item \textbf{文件排版:} 許多規範都會要求使用特定版型,使用文字編譯環境較能準確符合規定之版型,且能夠大範圍的自定義排定所需格式,並能不受之後更改而整體格式變形。
	\item \textbf{數學公式:} LaTex可以直接利用本身多元的模組套件加入、編輯數學公式,在數學推導過程能夠快速的輸入自己需要的內容即可且用於指令排版會相較符合所需的要求。
	\item \textbf{細部調整:} 在大型論文、報告中有多項文字、圖片、表格,需要調整細部時,要在好幾頁中找尋,而LaTeX可以分段章節進行編譯甚至可以與寫於主程式外,執行時在讀取進主程式執行,且可以分開執行,所以在執行時,不用一次讀取龐大的內容,完成後再進行合併處理大章節。
	\item \textbf{自動化功能:} LaTeX模組套件中其實就類似於一般高階程式語言的模組套件,以LaTeX的"tableofcontents"指令,可以自動生產目錄章節且會自動細分次章節、次次章節等,相較於Word等其他軟體方便。
\end{enumerate}
\par