\pagenumbering{roman}
\setcounter{page}{0}  %設定頁數
\chapter*{摘要}
\addcontentsline{toc}{chapter}{\sixteen 摘要}
\renewcommand{\baselinestretch}{2} %設定行距
\twelve 電腦的發明,使得學習的方式有了重大的改變,也使得e時代的崛起,在數位管理、平台、輔助工具上,有無限的方式進行運作,不知不覺中網際網路的模式已經漸漸地融入生活的一部份,並且慢慢地改變生活的習慣,雖然目前影響不是很大,但在未來將會成為生活的一部份。
\\
\par
\renewcommand{\baselinestretch}{1} %設定行距
\twelve 此專題所探討的是「網際內容管理系統在精密機械工程教學與研究上的應,用」,在精密機械上所需探討的知識層面很廣泛,有材料、製成、設計甚至包含跨領域的微處理器,這些都與精密機械息息相關,因此在眾多層面下,產生了許多資料和技術相關資訊,因此以「網際內容管理系統的技術」整合所有相關資訊。以目前重啟五專第一屆的精密機械科作為應用,使用Fossil作為應用的開發,探討如何提升教學與研究上的應用。\\
\par
\begin{center}
\twelve 關鍵字:網際內容管理、精密機械工程、Fossil SCM
\end{center}
\par