\chapter{總結}
\renewcommand{\baselinestretch}{10} %設定行距
%\section{前言}
\par
\renewcommand{\baselinestretch}{1} %設定行距
\twelve 本專題建立了一個網際內容管理系統的環境,並以精密機械工程科上本位發展為出發點,從教學與研究歷程的重要性作為研究動機,希望能夠將科系上學術課程及本位特色結合全體師生的合作,共同營造一個「精密科學習論壇」,且藉由這個教學論壇平台,能讓系上師生、畢業校友及業界學者在網際網路上交流與互動,達到提升課程教學與專題研究效益,且將「學術研究」、「實務運用」及「教學」整合一體,俾創造管理知識之價值及影響力,並在內容系統長期使用與管理下,可以當後續學生的資訊來源及研究之參考平台。
\\
\par
\renewcommand{\baselinestretch}{1} %設定行距
\twelve 根據現有文獻的進行探討,學習的方法及途徑已經逐步趨向數位化,想要獲取資訊是隨手可得的,但也因為大量資訊產生的世代,為了避免錯誤或毫無相關的資訊濫芋充數,而干擾視讀的判別及影響學習的成效,所以我們在精密科論壇上做到版次追朔、資料統整、分散式協同運作的功能;有別於現有的教學及論壇平台,此教學論壇平台能讓使用者在平台上去做知識管理與互動, 在蒐集資料的一個過程也能發表自己的看法和論點,透過與他人的意見討論及分析,不但能促使自我學習,也讓這個學習與互動的過程,在此教學平台上留下紀錄歷程,進而提升課程教學與研究的貢獻。 
\\
\par
\renewcommand{\baselinestretch}{1} %設定行距
\twelve 學習是永無止盡的,且學習的方式有千百種,然而網路的出現使學習不再受限,不但做到了國際間的資訊交流,也縮短了城鄉差距,共創了一個人們可以隨時隨地學習的環境;此平台的架設,是為了能讓未來精密機械工程科的師生們,更能了解系上的教學目標及成果,也希望在未來對於有興趣為系上貢獻的學弟妹們可以繼續共同開發這個環境,達到能透過網路影音自主學習的目標。
\par
