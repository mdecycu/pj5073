\chapter{未來規劃及目標}
\renewcommand{\baselinestretch}{10} %設定行距
%\section{前言}
\par
\renewcommand{\baselinestretch}{1} %設定行距
\twelve 本精密科教學論壇的建立,協助精密科學生共同交流、課程資訊、研究過程等相關問題,多數學生對於使用方向和論壇架構的概念仍然模糊,利用及參與的意願尚未建立,而精密科論壇最重要的就是需要透過學生們的共同參與,來促使資料的豐富性,因此宣傳及教學平台使用是後續推動論壇的課題。
\\
\par
\renewcommand{\baselinestretch}{1} %設定行距
\twelve 推廣精密科學習論壇有利條件:
\par
\begin{enumerate}
	\item 專屬精密科學生的課程資訊
	\begin{itemize}
		\item 學生能透過論壇尋找有關課程的進行方式,以及老師對這門課的要求。
		\item 透過考古題,能使學生準備考試時比較有準備的方向。
		\item 不斷地更能看到歷屆五專部學生五年的課程安排。
	\end{itemize}
	\item 精密科學生交流
	\begin{itemize}
		\item 老師的專業領域、專題製作方向。
		\item 系上相關資訊、證照題庫、考試標準。
		\item 新進學生需要注意什麼、生活瑣碎的事情、校園周邊的餐廳。
	\end{itemize}
	\item 經驗傳承
	\begin{itemize}
		\item 實習的相關資訊。
		\item 紀錄每一段時間學校對於五專部發展重點。
		\item 推動資源共構與合作學習,對於學生來說是絕佳方式。
		\item 未來成長的版圖和方向。
	\end{itemize}
\end{enumerate}
\par
\renewcommand{\baselinestretch}{1} %設定行距
\twelve 國中剛畢業的年紀就進入大學的生活,難免會有一些生活、課程銜接等問題,更何況學制相較於其他相同年齡層相對較稀少,因此未來走向及目標也相對較徬徨無助,若能從論壇中得到一絲絲的資訊,也會讓他們成長路程較充實及順暢,也能讓他們知道五專的優勢也不輸任何人,最終期望能秉持系上的座右銘,讓學生成為一個群體的精神,互相合作、互相勉勵,促使學生成為更有能裡的人,而回饋大家。
\par
